\documentclass[11pt]{article}

    \usepackage[breakable]{tcolorbox}
    \usepackage{parskip} % Stop auto-indenting (to mimic markdown behaviour)
    
    \usepackage{iftex}
    \ifPDFTeX
    	\usepackage[T1]{fontenc}
    	\usepackage{mathpazo}
    \else
    	\usepackage{fontspec}
    \fi

    % Basic figure setup, for now with no caption control since it's done
    % automatically by Pandoc (which extracts ![](path) syntax from Markdown).
    \usepackage{graphicx}
    % Maintain compatibility with old templates. Remove in nbconvert 6.0
    \let\Oldincludegraphics\includegraphics
    % Ensure that by default, figures have no caption (until we provide a
    % proper Figure object with a Caption API and a way to capture that
    % in the conversion process - todo).
    \usepackage{caption}
    \DeclareCaptionFormat{nocaption}{}
    \captionsetup{format=nocaption,aboveskip=0pt,belowskip=0pt}

    \usepackage[Export]{adjustbox} % Used to constrain images to a maximum size
    \adjustboxset{max size={0.9\linewidth}{0.9\paperheight}}
    \usepackage{float}
    \floatplacement{figure}{H} % forces figures to be placed at the correct location
    \usepackage{xcolor} % Allow colors to be defined
    \usepackage{enumerate} % Needed for markdown enumerations to work
    \usepackage{geometry} % Used to adjust the document margins
    \usepackage{amsmath} % Equations
    \usepackage{amssymb} % Equations
    \usepackage{textcomp} % defines textquotesingle
    % Hack from http://tex.stackexchange.com/a/47451/13684:
    \AtBeginDocument{%
        \def\PYZsq{\textquotesingle}% Upright quotes in Pygmentized code
    }
    \usepackage{upquote} % Upright quotes for verbatim code
    \usepackage{eurosym} % defines \euro
    \usepackage[mathletters]{ucs} % Extended unicode (utf-8) support
    \usepackage{fancyvrb} % verbatim replacement that allows latex
    \usepackage{grffile} % extends the file name processing of package graphics 
                         % to support a larger range
    \makeatletter % fix for grffile with XeLaTeX
    \def\Gread@@xetex#1{%
      \IfFileExists{"\Gin@base".bb}%
      {\Gread@eps{\Gin@base.bb}}%
      {\Gread@@xetex@aux#1}%
    }
    \makeatother

    % The hyperref package gives us a pdf with properly built
    % internal navigation ('pdf bookmarks' for the table of contents,
    % internal cross-reference links, web links for URLs, etc.)
    \usepackage{hyperref}
    % The default LaTeX title has an obnoxious amount of whitespace. By default,
    % titling removes some of it. It also provides customization options.
    \usepackage{titling}
    \usepackage{longtable} % longtable support required by pandoc >1.10
    \usepackage{booktabs}  % table support for pandoc > 1.12.2
    \usepackage[inline]{enumitem} % IRkernel/repr support (it uses the enumerate* environment)
    \usepackage[normalem]{ulem} % ulem is needed to support strikethroughs (\sout)
                                % normalem makes italics be italics, not underlines
    \usepackage{mathrsfs}
    

    
    % Colors for the hyperref package
    \definecolor{urlcolor}{rgb}{0,.145,.698}
    \definecolor{linkcolor}{rgb}{.71,0.21,0.01}
    \definecolor{citecolor}{rgb}{.12,.54,.11}

    % ANSI colors
    \definecolor{ansi-black}{HTML}{3E424D}
    \definecolor{ansi-black-intense}{HTML}{282C36}
    \definecolor{ansi-red}{HTML}{E75C58}
    \definecolor{ansi-red-intense}{HTML}{B22B31}
    \definecolor{ansi-green}{HTML}{00A250}
    \definecolor{ansi-green-intense}{HTML}{007427}
    \definecolor{ansi-yellow}{HTML}{DDB62B}
    \definecolor{ansi-yellow-intense}{HTML}{B27D12}
    \definecolor{ansi-blue}{HTML}{208FFB}
    \definecolor{ansi-blue-intense}{HTML}{0065CA}
    \definecolor{ansi-magenta}{HTML}{D160C4}
    \definecolor{ansi-magenta-intense}{HTML}{A03196}
    \definecolor{ansi-cyan}{HTML}{60C6C8}
    \definecolor{ansi-cyan-intense}{HTML}{258F8F}
    \definecolor{ansi-white}{HTML}{C5C1B4}
    \definecolor{ansi-white-intense}{HTML}{A1A6B2}
    \definecolor{ansi-default-inverse-fg}{HTML}{FFFFFF}
    \definecolor{ansi-default-inverse-bg}{HTML}{000000}

    % commands and environments needed by pandoc snippets
    % extracted from the output of `pandoc -s`
    \providecommand{\tightlist}{%
      \setlength{\itemsep}{0pt}\setlength{\parskip}{0pt}}
    \DefineVerbatimEnvironment{Highlighting}{Verbatim}{commandchars=\\\{\}}
    % Add ',fontsize=\small' for more characters per line
    \newenvironment{Shaded}{}{}
    \newcommand{\KeywordTok}[1]{\textcolor[rgb]{0.00,0.44,0.13}{\textbf{{#1}}}}
    \newcommand{\DataTypeTok}[1]{\textcolor[rgb]{0.56,0.13,0.00}{{#1}}}
    \newcommand{\DecValTok}[1]{\textcolor[rgb]{0.25,0.63,0.44}{{#1}}}
    \newcommand{\BaseNTok}[1]{\textcolor[rgb]{0.25,0.63,0.44}{{#1}}}
    \newcommand{\FloatTok}[1]{\textcolor[rgb]{0.25,0.63,0.44}{{#1}}}
    \newcommand{\CharTok}[1]{\textcolor[rgb]{0.25,0.44,0.63}{{#1}}}
    \newcommand{\StringTok}[1]{\textcolor[rgb]{0.25,0.44,0.63}{{#1}}}
    \newcommand{\CommentTok}[1]{\textcolor[rgb]{0.38,0.63,0.69}{\textit{{#1}}}}
    \newcommand{\OtherTok}[1]{\textcolor[rgb]{0.00,0.44,0.13}{{#1}}}
    \newcommand{\AlertTok}[1]{\textcolor[rgb]{1.00,0.00,0.00}{\textbf{{#1}}}}
    \newcommand{\FunctionTok}[1]{\textcolor[rgb]{0.02,0.16,0.49}{{#1}}}
    \newcommand{\RegionMarkerTok}[1]{{#1}}
    \newcommand{\ErrorTok}[1]{\textcolor[rgb]{1.00,0.00,0.00}{\textbf{{#1}}}}
    \newcommand{\NormalTok}[1]{{#1}}
    
    % Additional commands for more recent versions of Pandoc
    \newcommand{\ConstantTok}[1]{\textcolor[rgb]{0.53,0.00,0.00}{{#1}}}
    \newcommand{\SpecialCharTok}[1]{\textcolor[rgb]{0.25,0.44,0.63}{{#1}}}
    \newcommand{\VerbatimStringTok}[1]{\textcolor[rgb]{0.25,0.44,0.63}{{#1}}}
    \newcommand{\SpecialStringTok}[1]{\textcolor[rgb]{0.73,0.40,0.53}{{#1}}}
    \newcommand{\ImportTok}[1]{{#1}}
    \newcommand{\DocumentationTok}[1]{\textcolor[rgb]{0.73,0.13,0.13}{\textit{{#1}}}}
    \newcommand{\AnnotationTok}[1]{\textcolor[rgb]{0.38,0.63,0.69}{\textbf{\textit{{#1}}}}}
    \newcommand{\CommentVarTok}[1]{\textcolor[rgb]{0.38,0.63,0.69}{\textbf{\textit{{#1}}}}}
    \newcommand{\VariableTok}[1]{\textcolor[rgb]{0.10,0.09,0.49}{{#1}}}
    \newcommand{\ControlFlowTok}[1]{\textcolor[rgb]{0.00,0.44,0.13}{\textbf{{#1}}}}
    \newcommand{\OperatorTok}[1]{\textcolor[rgb]{0.40,0.40,0.40}{{#1}}}
    \newcommand{\BuiltInTok}[1]{{#1}}
    \newcommand{\ExtensionTok}[1]{{#1}}
    \newcommand{\PreprocessorTok}[1]{\textcolor[rgb]{0.74,0.48,0.00}{{#1}}}
    \newcommand{\AttributeTok}[1]{\textcolor[rgb]{0.49,0.56,0.16}{{#1}}}
    \newcommand{\InformationTok}[1]{\textcolor[rgb]{0.38,0.63,0.69}{\textbf{\textit{{#1}}}}}
    \newcommand{\WarningTok}[1]{\textcolor[rgb]{0.38,0.63,0.69}{\textbf{\textit{{#1}}}}}
    
    
    % Define a nice break command that doesn't care if a line doesn't already
    % exist.
    \def\br{\hspace*{\fill} \\* }
    % Math Jax compatibility definitions
    \def\gt{>}
    \def\lt{<}
    \let\Oldtex\TeX
    \let\Oldlatex\LaTeX
    \renewcommand{\TeX}{\textrm{\Oldtex}}
    \renewcommand{\LaTeX}{\textrm{\Oldlatex}}
    % Document parameters
    % Document title
    \title{Instalación y configuración inicial de Anaconda}
    
    
    
    
    
% Pygments definitions
\makeatletter
\def\PY@reset{\let\PY@it=\relax \let\PY@bf=\relax%
    \let\PY@ul=\relax \let\PY@tc=\relax%
    \let\PY@bc=\relax \let\PY@ff=\relax}
\def\PY@tok#1{\csname PY@tok@#1\endcsname}
\def\PY@toks#1+{\ifx\relax#1\empty\else%
    \PY@tok{#1}\expandafter\PY@toks\fi}
\def\PY@do#1{\PY@bc{\PY@tc{\PY@ul{%
    \PY@it{\PY@bf{\PY@ff{#1}}}}}}}
\def\PY#1#2{\PY@reset\PY@toks#1+\relax+\PY@do{#2}}

\expandafter\def\csname PY@tok@w\endcsname{\def\PY@tc##1{\textcolor[rgb]{0.73,0.73,0.73}{##1}}}
\expandafter\def\csname PY@tok@c\endcsname{\let\PY@it=\textit\def\PY@tc##1{\textcolor[rgb]{0.25,0.50,0.50}{##1}}}
\expandafter\def\csname PY@tok@cp\endcsname{\def\PY@tc##1{\textcolor[rgb]{0.74,0.48,0.00}{##1}}}
\expandafter\def\csname PY@tok@k\endcsname{\let\PY@bf=\textbf\def\PY@tc##1{\textcolor[rgb]{0.00,0.50,0.00}{##1}}}
\expandafter\def\csname PY@tok@kp\endcsname{\def\PY@tc##1{\textcolor[rgb]{0.00,0.50,0.00}{##1}}}
\expandafter\def\csname PY@tok@kt\endcsname{\def\PY@tc##1{\textcolor[rgb]{0.69,0.00,0.25}{##1}}}
\expandafter\def\csname PY@tok@o\endcsname{\def\PY@tc##1{\textcolor[rgb]{0.40,0.40,0.40}{##1}}}
\expandafter\def\csname PY@tok@ow\endcsname{\let\PY@bf=\textbf\def\PY@tc##1{\textcolor[rgb]{0.67,0.13,1.00}{##1}}}
\expandafter\def\csname PY@tok@nb\endcsname{\def\PY@tc##1{\textcolor[rgb]{0.00,0.50,0.00}{##1}}}
\expandafter\def\csname PY@tok@nf\endcsname{\def\PY@tc##1{\textcolor[rgb]{0.00,0.00,1.00}{##1}}}
\expandafter\def\csname PY@tok@nc\endcsname{\let\PY@bf=\textbf\def\PY@tc##1{\textcolor[rgb]{0.00,0.00,1.00}{##1}}}
\expandafter\def\csname PY@tok@nn\endcsname{\let\PY@bf=\textbf\def\PY@tc##1{\textcolor[rgb]{0.00,0.00,1.00}{##1}}}
\expandafter\def\csname PY@tok@ne\endcsname{\let\PY@bf=\textbf\def\PY@tc##1{\textcolor[rgb]{0.82,0.25,0.23}{##1}}}
\expandafter\def\csname PY@tok@nv\endcsname{\def\PY@tc##1{\textcolor[rgb]{0.10,0.09,0.49}{##1}}}
\expandafter\def\csname PY@tok@no\endcsname{\def\PY@tc##1{\textcolor[rgb]{0.53,0.00,0.00}{##1}}}
\expandafter\def\csname PY@tok@nl\endcsname{\def\PY@tc##1{\textcolor[rgb]{0.63,0.63,0.00}{##1}}}
\expandafter\def\csname PY@tok@ni\endcsname{\let\PY@bf=\textbf\def\PY@tc##1{\textcolor[rgb]{0.60,0.60,0.60}{##1}}}
\expandafter\def\csname PY@tok@na\endcsname{\def\PY@tc##1{\textcolor[rgb]{0.49,0.56,0.16}{##1}}}
\expandafter\def\csname PY@tok@nt\endcsname{\let\PY@bf=\textbf\def\PY@tc##1{\textcolor[rgb]{0.00,0.50,0.00}{##1}}}
\expandafter\def\csname PY@tok@nd\endcsname{\def\PY@tc##1{\textcolor[rgb]{0.67,0.13,1.00}{##1}}}
\expandafter\def\csname PY@tok@s\endcsname{\def\PY@tc##1{\textcolor[rgb]{0.73,0.13,0.13}{##1}}}
\expandafter\def\csname PY@tok@sd\endcsname{\let\PY@it=\textit\def\PY@tc##1{\textcolor[rgb]{0.73,0.13,0.13}{##1}}}
\expandafter\def\csname PY@tok@si\endcsname{\let\PY@bf=\textbf\def\PY@tc##1{\textcolor[rgb]{0.73,0.40,0.53}{##1}}}
\expandafter\def\csname PY@tok@se\endcsname{\let\PY@bf=\textbf\def\PY@tc##1{\textcolor[rgb]{0.73,0.40,0.13}{##1}}}
\expandafter\def\csname PY@tok@sr\endcsname{\def\PY@tc##1{\textcolor[rgb]{0.73,0.40,0.53}{##1}}}
\expandafter\def\csname PY@tok@ss\endcsname{\def\PY@tc##1{\textcolor[rgb]{0.10,0.09,0.49}{##1}}}
\expandafter\def\csname PY@tok@sx\endcsname{\def\PY@tc##1{\textcolor[rgb]{0.00,0.50,0.00}{##1}}}
\expandafter\def\csname PY@tok@m\endcsname{\def\PY@tc##1{\textcolor[rgb]{0.40,0.40,0.40}{##1}}}
\expandafter\def\csname PY@tok@gh\endcsname{\let\PY@bf=\textbf\def\PY@tc##1{\textcolor[rgb]{0.00,0.00,0.50}{##1}}}
\expandafter\def\csname PY@tok@gu\endcsname{\let\PY@bf=\textbf\def\PY@tc##1{\textcolor[rgb]{0.50,0.00,0.50}{##1}}}
\expandafter\def\csname PY@tok@gd\endcsname{\def\PY@tc##1{\textcolor[rgb]{0.63,0.00,0.00}{##1}}}
\expandafter\def\csname PY@tok@gi\endcsname{\def\PY@tc##1{\textcolor[rgb]{0.00,0.63,0.00}{##1}}}
\expandafter\def\csname PY@tok@gr\endcsname{\def\PY@tc##1{\textcolor[rgb]{1.00,0.00,0.00}{##1}}}
\expandafter\def\csname PY@tok@ge\endcsname{\let\PY@it=\textit}
\expandafter\def\csname PY@tok@gs\endcsname{\let\PY@bf=\textbf}
\expandafter\def\csname PY@tok@gp\endcsname{\let\PY@bf=\textbf\def\PY@tc##1{\textcolor[rgb]{0.00,0.00,0.50}{##1}}}
\expandafter\def\csname PY@tok@go\endcsname{\def\PY@tc##1{\textcolor[rgb]{0.53,0.53,0.53}{##1}}}
\expandafter\def\csname PY@tok@gt\endcsname{\def\PY@tc##1{\textcolor[rgb]{0.00,0.27,0.87}{##1}}}
\expandafter\def\csname PY@tok@err\endcsname{\def\PY@bc##1{\setlength{\fboxsep}{0pt}\fcolorbox[rgb]{1.00,0.00,0.00}{1,1,1}{\strut ##1}}}
\expandafter\def\csname PY@tok@kc\endcsname{\let\PY@bf=\textbf\def\PY@tc##1{\textcolor[rgb]{0.00,0.50,0.00}{##1}}}
\expandafter\def\csname PY@tok@kd\endcsname{\let\PY@bf=\textbf\def\PY@tc##1{\textcolor[rgb]{0.00,0.50,0.00}{##1}}}
\expandafter\def\csname PY@tok@kn\endcsname{\let\PY@bf=\textbf\def\PY@tc##1{\textcolor[rgb]{0.00,0.50,0.00}{##1}}}
\expandafter\def\csname PY@tok@kr\endcsname{\let\PY@bf=\textbf\def\PY@tc##1{\textcolor[rgb]{0.00,0.50,0.00}{##1}}}
\expandafter\def\csname PY@tok@bp\endcsname{\def\PY@tc##1{\textcolor[rgb]{0.00,0.50,0.00}{##1}}}
\expandafter\def\csname PY@tok@fm\endcsname{\def\PY@tc##1{\textcolor[rgb]{0.00,0.00,1.00}{##1}}}
\expandafter\def\csname PY@tok@vc\endcsname{\def\PY@tc##1{\textcolor[rgb]{0.10,0.09,0.49}{##1}}}
\expandafter\def\csname PY@tok@vg\endcsname{\def\PY@tc##1{\textcolor[rgb]{0.10,0.09,0.49}{##1}}}
\expandafter\def\csname PY@tok@vi\endcsname{\def\PY@tc##1{\textcolor[rgb]{0.10,0.09,0.49}{##1}}}
\expandafter\def\csname PY@tok@vm\endcsname{\def\PY@tc##1{\textcolor[rgb]{0.10,0.09,0.49}{##1}}}
\expandafter\def\csname PY@tok@sa\endcsname{\def\PY@tc##1{\textcolor[rgb]{0.73,0.13,0.13}{##1}}}
\expandafter\def\csname PY@tok@sb\endcsname{\def\PY@tc##1{\textcolor[rgb]{0.73,0.13,0.13}{##1}}}
\expandafter\def\csname PY@tok@sc\endcsname{\def\PY@tc##1{\textcolor[rgb]{0.73,0.13,0.13}{##1}}}
\expandafter\def\csname PY@tok@dl\endcsname{\def\PY@tc##1{\textcolor[rgb]{0.73,0.13,0.13}{##1}}}
\expandafter\def\csname PY@tok@s2\endcsname{\def\PY@tc##1{\textcolor[rgb]{0.73,0.13,0.13}{##1}}}
\expandafter\def\csname PY@tok@sh\endcsname{\def\PY@tc##1{\textcolor[rgb]{0.73,0.13,0.13}{##1}}}
\expandafter\def\csname PY@tok@s1\endcsname{\def\PY@tc##1{\textcolor[rgb]{0.73,0.13,0.13}{##1}}}
\expandafter\def\csname PY@tok@mb\endcsname{\def\PY@tc##1{\textcolor[rgb]{0.40,0.40,0.40}{##1}}}
\expandafter\def\csname PY@tok@mf\endcsname{\def\PY@tc##1{\textcolor[rgb]{0.40,0.40,0.40}{##1}}}
\expandafter\def\csname PY@tok@mh\endcsname{\def\PY@tc##1{\textcolor[rgb]{0.40,0.40,0.40}{##1}}}
\expandafter\def\csname PY@tok@mi\endcsname{\def\PY@tc##1{\textcolor[rgb]{0.40,0.40,0.40}{##1}}}
\expandafter\def\csname PY@tok@il\endcsname{\def\PY@tc##1{\textcolor[rgb]{0.40,0.40,0.40}{##1}}}
\expandafter\def\csname PY@tok@mo\endcsname{\def\PY@tc##1{\textcolor[rgb]{0.40,0.40,0.40}{##1}}}
\expandafter\def\csname PY@tok@ch\endcsname{\let\PY@it=\textit\def\PY@tc##1{\textcolor[rgb]{0.25,0.50,0.50}{##1}}}
\expandafter\def\csname PY@tok@cm\endcsname{\let\PY@it=\textit\def\PY@tc##1{\textcolor[rgb]{0.25,0.50,0.50}{##1}}}
\expandafter\def\csname PY@tok@cpf\endcsname{\let\PY@it=\textit\def\PY@tc##1{\textcolor[rgb]{0.25,0.50,0.50}{##1}}}
\expandafter\def\csname PY@tok@c1\endcsname{\let\PY@it=\textit\def\PY@tc##1{\textcolor[rgb]{0.25,0.50,0.50}{##1}}}
\expandafter\def\csname PY@tok@cs\endcsname{\let\PY@it=\textit\def\PY@tc##1{\textcolor[rgb]{0.25,0.50,0.50}{##1}}}

\def\PYZbs{\char`\\}
\def\PYZus{\char`\_}
\def\PYZob{\char`\{}
\def\PYZcb{\char`\}}
\def\PYZca{\char`\^}
\def\PYZam{\char`\&}
\def\PYZlt{\char`\<}
\def\PYZgt{\char`\>}
\def\PYZsh{\char`\#}
\def\PYZpc{\char`\%}
\def\PYZdl{\char`\$}
\def\PYZhy{\char`\-}
\def\PYZsq{\char`\'}
\def\PYZdq{\char`\"}
\def\PYZti{\char`\~}
% for compatibility with earlier versions
\def\PYZat{@}
\def\PYZlb{[}
\def\PYZrb{]}
\makeatother


    % For linebreaks inside Verbatim environment from package fancyvrb. 
    \makeatletter
        \newbox\Wrappedcontinuationbox 
        \newbox\Wrappedvisiblespacebox 
        \newcommand*\Wrappedvisiblespace {\textcolor{red}{\textvisiblespace}} 
        \newcommand*\Wrappedcontinuationsymbol {\textcolor{red}{\llap{\tiny$\m@th\hookrightarrow$}}} 
        \newcommand*\Wrappedcontinuationindent {3ex } 
        \newcommand*\Wrappedafterbreak {\kern\Wrappedcontinuationindent\copy\Wrappedcontinuationbox} 
        % Take advantage of the already applied Pygments mark-up to insert 
        % potential linebreaks for TeX processing. 
        %        {, <, #, %, $, ' and ": go to next line. 
        %        _, }, ^, &, >, - and ~: stay at end of broken line. 
        % Use of \textquotesingle for straight quote. 
        \newcommand*\Wrappedbreaksatspecials {% 
            \def\PYGZus{\discretionary{\char`\_}{\Wrappedafterbreak}{\char`\_}}% 
            \def\PYGZob{\discretionary{}{\Wrappedafterbreak\char`\{}{\char`\{}}% 
            \def\PYGZcb{\discretionary{\char`\}}{\Wrappedafterbreak}{\char`\}}}% 
            \def\PYGZca{\discretionary{\char`\^}{\Wrappedafterbreak}{\char`\^}}% 
            \def\PYGZam{\discretionary{\char`\&}{\Wrappedafterbreak}{\char`\&}}% 
            \def\PYGZlt{\discretionary{}{\Wrappedafterbreak\char`\<}{\char`\<}}% 
            \def\PYGZgt{\discretionary{\char`\>}{\Wrappedafterbreak}{\char`\>}}% 
            \def\PYGZsh{\discretionary{}{\Wrappedafterbreak\char`\#}{\char`\#}}% 
            \def\PYGZpc{\discretionary{}{\Wrappedafterbreak\char`\%}{\char`\%}}% 
            \def\PYGZdl{\discretionary{}{\Wrappedafterbreak\char`\$}{\char`\$}}% 
            \def\PYGZhy{\discretionary{\char`\-}{\Wrappedafterbreak}{\char`\-}}% 
            \def\PYGZsq{\discretionary{}{\Wrappedafterbreak\textquotesingle}{\textquotesingle}}% 
            \def\PYGZdq{\discretionary{}{\Wrappedafterbreak\char`\"}{\char`\"}}% 
            \def\PYGZti{\discretionary{\char`\~}{\Wrappedafterbreak}{\char`\~}}% 
        } 
        % Some characters . , ; ? ! / are not pygmentized. 
        % This macro makes them "active" and they will insert potential linebreaks 
        \newcommand*\Wrappedbreaksatpunct {% 
            \lccode`\~`\.\lowercase{\def~}{\discretionary{\hbox{\char`\.}}{\Wrappedafterbreak}{\hbox{\char`\.}}}% 
            \lccode`\~`\,\lowercase{\def~}{\discretionary{\hbox{\char`\,}}{\Wrappedafterbreak}{\hbox{\char`\,}}}% 
            \lccode`\~`\;\lowercase{\def~}{\discretionary{\hbox{\char`\;}}{\Wrappedafterbreak}{\hbox{\char`\;}}}% 
            \lccode`\~`\:\lowercase{\def~}{\discretionary{\hbox{\char`\:}}{\Wrappedafterbreak}{\hbox{\char`\:}}}% 
            \lccode`\~`\?\lowercase{\def~}{\discretionary{\hbox{\char`\?}}{\Wrappedafterbreak}{\hbox{\char`\?}}}% 
            \lccode`\~`\!\lowercase{\def~}{\discretionary{\hbox{\char`\!}}{\Wrappedafterbreak}{\hbox{\char`\!}}}% 
            \lccode`\~`\/\lowercase{\def~}{\discretionary{\hbox{\char`\/}}{\Wrappedafterbreak}{\hbox{\char`\/}}}% 
            \catcode`\.\active
            \catcode`\,\active 
            \catcode`\;\active
            \catcode`\:\active
            \catcode`\?\active
            \catcode`\!\active
            \catcode`\/\active 
            \lccode`\~`\~ 	
        }
    \makeatother

    \let\OriginalVerbatim=\Verbatim
    \makeatletter
    \renewcommand{\Verbatim}[1][1]{%
        %\parskip\z@skip
        \sbox\Wrappedcontinuationbox {\Wrappedcontinuationsymbol}%
        \sbox\Wrappedvisiblespacebox {\FV@SetupFont\Wrappedvisiblespace}%
        \def\FancyVerbFormatLine ##1{\hsize\linewidth
            \vtop{\raggedright\hyphenpenalty\z@\exhyphenpenalty\z@
                \doublehyphendemerits\z@\finalhyphendemerits\z@
                \strut ##1\strut}%
        }%
        % If the linebreak is at a space, the latter will be displayed as visible
        % space at end of first line, and a continuation symbol starts next line.
        % Stretch/shrink are however usually zero for typewriter font.
        \def\FV@Space {%
            \nobreak\hskip\z@ plus\fontdimen3\font minus\fontdimen4\font
            \discretionary{\copy\Wrappedvisiblespacebox}{\Wrappedafterbreak}
            {\kern\fontdimen2\font}%
        }%
        
        % Allow breaks at special characters using \PYG... macros.
        \Wrappedbreaksatspecials
        % Breaks at punctuation characters . , ; ? ! and / need catcode=\active 	
        \OriginalVerbatim[#1,codes*=\Wrappedbreaksatpunct]%
    }
    \makeatother

    % Exact colors from NB
    \definecolor{incolor}{HTML}{303F9F}
    \definecolor{outcolor}{HTML}{D84315}
    \definecolor{cellborder}{HTML}{CFCFCF}
    \definecolor{cellbackground}{HTML}{F7F7F7}
    
    % prompt
    \makeatletter
    \newcommand{\boxspacing}{\kern\kvtcb@left@rule\kern\kvtcb@boxsep}
    \makeatother
    \newcommand{\prompt}[4]{
        \ttfamily\llap{{\color{#2}[#3]:\hspace{3pt}#4}}\vspace{-\baselineskip}
    }
    

    
    % Prevent overflowing lines due to hard-to-break entities
    \sloppy 
    % Setup hyperref package
    \hypersetup{
      breaklinks=true,  % so long urls are correctly broken across lines
      colorlinks=true,
      urlcolor=urlcolor,
      linkcolor=linkcolor,
      citecolor=citecolor,
      }
    % Slightly bigger margins than the latex defaults
    
    \geometry{verbose,tmargin=1in,bmargin=1in,lmargin=1in,rmargin=1in}
    
    

\begin{document}
    
    \maketitle
    
    

    
    \hypertarget{muxe9todos-cuantitavos-en-gestiuxf3n-2020}{%
\section{Métodos Cuantitavos en Gestión
2020}\label{muxe9todos-cuantitavos-en-gestiuxf3n-2020}}

\textbf{Universidad Externado de Colombia}

\emph{Facultad de Administración de Empresas}

\emph{2020-1}

Las nuevas herramientas tecnológicas y digitales le han dado un vuelco
total a nuestra sociedad. Las facilidades en la comunicación y la
superconectividad entre todo el mundo hace que nuestros intereses,
gustos y experiencias sean registradas, descritas, fotografiadas y por
eso mismo monitoreadas. Así, la toma de decisiones de cualquier
organización se fundamenta en sus datos registrados y en la capacidad de
procesarlos, por lo tanto, el ejercicio del gerente actual es acertar en
sus decisiones de acuerdo con el constante y abrumador cambio en el
mundo de hoy.

¿Cómo lograr una decisión acertada en un mundo cambiante? No tenemos la
respuesta, pero sí sabemos que las herramientas cuantitativas son
poderosas para este cometido, de hecho, las grandes iniciativas se
soportan sobre bellas construcciones matemáticas, bien sean algoritmos,
modelos, métodos, entre otros. La idea de este curso es introducirlos en
el uso practico de estas herramientas con ayuda de la tecnología. Nos
separamos un poco de la noción tradicional de clase de matemáticas para
apreciar la gran magnitud del encanto en esta ciencia y reconocerla como
la principal aliada en la toma de nuestras decisiones.

El siguiente video del Matemático galardonado con la medalla Fields en
2010 \emph{Cedric Villani} habla un poco de esta sensualidad de las
matemáticas y así mismo desmiente la creencia generalizada que confunde
la Ciencia Matemática con el Cálculo.

\href{https://www.ted.com/talks/cedric_villani_what_s_so_sexy_about_math}{Enlace
del video}

    \begin{tcolorbox}[breakable, size=fbox, boxrule=1pt, pad at break*=1mm,colback=cellbackground, colframe=cellborder]
\prompt{In}{incolor}{1}{\boxspacing}
\begin{Verbatim}[commandchars=\\\{\}]
\PY{k+kn}{from} \PY{n+nn}{IPython}\PY{n+nn}{.}\PY{n+nn}{display} \PY{k+kn}{import} \PY{n}{HTML}
\end{Verbatim}
\end{tcolorbox}

    \begin{tcolorbox}[breakable, size=fbox, boxrule=1pt, pad at break*=1mm,colback=cellbackground, colframe=cellborder]
\prompt{In}{incolor}{2}{\boxspacing}
\begin{Verbatim}[commandchars=\\\{\}]
\PY{n}{HTML}\PY{p}{(}\PY{l+s+s1}{\PYZsq{}}\PY{l+s+s1}{\PYZlt{}div style=}\PY{l+s+s1}{\PYZdq{}}\PY{l+s+s1}{max\PYZhy{}width:854px}\PY{l+s+s1}{\PYZdq{}}\PY{l+s+s1}{\PYZgt{}\PYZlt{}div style=}\PY{l+s+s1}{\PYZdq{}}\PY{l+s+s1}{position:relative;height:0;padding\PYZhy{}bottom:100}\PY{l+s+s1}{\PYZpc{}}\PY{l+s+s1}{\PYZdq{}}\PY{l+s+s1}{\PYZgt{}\PYZlt{}iframe src=}\PY{l+s+s1}{\PYZdq{}}\PY{l+s+s1}{https://embed.ted.com/talks/cedric\PYZus{}villani\PYZus{}what\PYZus{}s\PYZus{}so\PYZus{}sexy\PYZus{}about\PYZus{}math}\PY{l+s+s1}{\PYZdq{}}\PY{l+s+s1}{ width=}\PY{l+s+s1}{\PYZdq{}}\PY{l+s+s1}{854}\PY{l+s+s1}{\PYZdq{}}\PY{l+s+s1}{ height=}\PY{l+s+s1}{\PYZdq{}}\PY{l+s+s1}{480}\PY{l+s+s1}{\PYZdq{}}\PY{l+s+s1}{ style=}\PY{l+s+s1}{\PYZdq{}}\PY{l+s+s1}{position:absolute;left:0;top:0;width:100}\PY{l+s+s1}{\PYZpc{}}\PY{l+s+s1}{;height:100}\PY{l+s+s1}{\PYZpc{}}\PY{l+s+s1}{\PYZdq{}}\PY{l+s+s1}{ frameborder=}\PY{l+s+s1}{\PYZdq{}}\PY{l+s+s1}{0}\PY{l+s+s1}{\PYZdq{}}\PY{l+s+s1}{ scrolling=}\PY{l+s+s1}{\PYZdq{}}\PY{l+s+s1}{no}\PY{l+s+s1}{\PYZdq{}}\PY{l+s+s1}{ allowfullscreen\PYZgt{}\PYZlt{}/iframe\PYZgt{}\PYZlt{}/div\PYZgt{}\PYZlt{}/div\PYZgt{}}\PY{l+s+s1}{\PYZsq{}}\PY{p}{)}
\end{Verbatim}
\end{tcolorbox}

            \begin{tcolorbox}[breakable, size=fbox, boxrule=.5pt, pad at break*=1mm, opacityfill=0]
\prompt{Out}{outcolor}{2}{\boxspacing}
\begin{Verbatim}[commandchars=\\\{\}]
<IPython.core.display.HTML object>
\end{Verbatim}
\end{tcolorbox}
        
    Así mismo, Conrad Wolfram nos sugiere algunos aspectos importantes a
tener en cuenta para recrear las metodologías usadas en la enseñanza de
las matemáticas:

\href{https://www.ted.com/talks/conrad_wolfram_teaching_kids_real_math_with_computers/transcript?language=es}{Enlace
del Video}

    \begin{tcolorbox}[breakable, size=fbox, boxrule=1pt, pad at break*=1mm,colback=cellbackground, colframe=cellborder]
\prompt{In}{incolor}{3}{\boxspacing}
\begin{Verbatim}[commandchars=\\\{\}]
\PY{n}{HTML}\PY{p}{(}\PY{l+s+s1}{\PYZsq{}}\PY{l+s+s1}{\PYZlt{}div style=}\PY{l+s+s1}{\PYZdq{}}\PY{l+s+s1}{max\PYZhy{}width:854px}\PY{l+s+s1}{\PYZdq{}}\PY{l+s+s1}{\PYZgt{}\PYZlt{}div style=}\PY{l+s+s1}{\PYZdq{}}\PY{l+s+s1}{position:relative;height:0;padding\PYZhy{}bottom:56.25}\PY{l+s+s1}{\PYZpc{}}\PY{l+s+s1}{\PYZdq{}}\PY{l+s+s1}{\PYZgt{}\PYZlt{}iframe src=}\PY{l+s+s1}{\PYZdq{}}\PY{l+s+s1}{https://embed.ted.com/talks/lang/es/conrad\PYZus{}wolfram\PYZus{}teaching\PYZus{}kids\PYZus{}real\PYZus{}math\PYZus{}with\PYZus{}computers}\PY{l+s+s1}{\PYZdq{}}\PY{l+s+s1}{ width=}\PY{l+s+s1}{\PYZdq{}}\PY{l+s+s1}{854}\PY{l+s+s1}{\PYZdq{}}\PY{l+s+s1}{ height=}\PY{l+s+s1}{\PYZdq{}}\PY{l+s+s1}{480}\PY{l+s+s1}{\PYZdq{}}\PY{l+s+s1}{ style=}\PY{l+s+s1}{\PYZdq{}}\PY{l+s+s1}{position:absolute;left:0;top:0;width:100}\PY{l+s+s1}{\PYZpc{}}\PY{l+s+s1}{;height:100}\PY{l+s+s1}{\PYZpc{}}\PY{l+s+s1}{\PYZdq{}}\PY{l+s+s1}{ frameborder=}\PY{l+s+s1}{\PYZdq{}}\PY{l+s+s1}{0}\PY{l+s+s1}{\PYZdq{}}\PY{l+s+s1}{ scrolling=}\PY{l+s+s1}{\PYZdq{}}\PY{l+s+s1}{no}\PY{l+s+s1}{\PYZdq{}}\PY{l+s+s1}{ allowfullscreen\PYZgt{}\PYZlt{}/iframe\PYZgt{}\PYZlt{}/div\PYZgt{}\PYZlt{}/div\PYZgt{}}\PY{l+s+s1}{\PYZsq{}}\PY{p}{)}
\end{Verbatim}
\end{tcolorbox}

            \begin{tcolorbox}[breakable, size=fbox, boxrule=.5pt, pad at break*=1mm, opacityfill=0]
\prompt{Out}{outcolor}{3}{\boxspacing}
\begin{Verbatim}[commandchars=\\\{\}]
<IPython.core.display.HTML object>
\end{Verbatim}
\end{tcolorbox}
        
    \hypertarget{uso-de-jupyter}{%
\section{Uso de Jupyter}\label{uso-de-jupyter}}

Antes de iniciar nuestro curso es necesario instalar una herramienta que
calcule por nosotros. Usaremos Python, un lenguaje de programación con
un amplio espectro de utilidades que va desde la solución de simples
cálculos matemáticos hasta el desarrollo de aplicaciones web. La
distribución que utilizaremos es
\href{https://www.anaconda.com/distribution/}{Anaconda}, una plataforma
especializada en el aprendizaje automático y la ciencia de los datos y
de la plataforma usaremos Jupyter, una aplicación web que permite crear
cuadernos interacivos, documentos web editados con texto enriquecido y
celdas de ejecución de código.

La posibilidad de ejecutar código nos será muy útil para definir
algoritmos, procesar datos y visualizar resultados en un mismo
documento.

    \hypertarget{instalaciuxf3n-y-primer-cuaderno}{%
\subsection{Instalación y primer
cuaderno}\label{instalaciuxf3n-y-primer-cuaderno}}

Sigue los siguientes pasos para instalar jupyter en tu pc:

\begin{enumerate}
\def\labelenumi{\arabic{enumi}.}
\tightlist
\item
  Descargar Anaconda de (https://www.anaconda.com/distribution/),
  verifica que la descarga corresponda a Python 3.7 y a las condiciones
  específicas de tu equipo (Windows x64, Windows x32, macOS o Linux)
\end{enumerate}

\begin{enumerate}
\def\labelenumi{\arabic{enumi}.}
\setcounter{enumi}{1}
\tightlist
\item
  Despúes de instalar Anaconda abrelo y lanza el aplicativo de Jupyter:
\end{enumerate}

\begin{enumerate}
\def\labelenumi{\arabic{enumi}.}
\setcounter{enumi}{2}
\tightlist
\item
  Jupyter abrirá una pestaña de tu navegador predeterminado, verás algo
  así:
\end{enumerate}

\begin{enumerate}
\def\labelenumi{\arabic{enumi}.}
\setcounter{enumi}{3}
\tightlist
\item
  En el boton NEW puedes crear un archivo nuevo en Python 3:
\end{enumerate}

\begin{enumerate}
\def\labelenumi{\arabic{enumi}.}
\setcounter{enumi}{4}
\tightlist
\item
  Veras un panel como el siguiente, selecciona en la casilla tipo de
  celda selecciona Markdown, para titular el cuaderno que vas a crear,
  tambien puedes oprimir {[}Esc{]}+{[}M{]}:
\end{enumerate}

\begin{enumerate}
\def\labelenumi{\arabic{enumi}.}
\setcounter{enumi}{5}
\tightlist
\item
  Después de cambiar el tipo de celda usa el formato Markdown para hacer
  encabezados del documento, escribes \textbf{\emph{\# Título}} para el
  encabezado principal y \textbf{\emph{\#\# Subtítulo}} para el
  encabezado secundario:
\end{enumerate}

\begin{enumerate}
\def\labelenumi{\arabic{enumi}.}
\setcounter{enumi}{6}
\tightlist
\item
  Después de escribir oprimes {[}Shift{]}+{[}Enter{]} para ejecutar la
  celda y visualizar el resultado final de tu escrito, obtienes:
\end{enumerate}

\begin{enumerate}
\def\labelenumi{\arabic{enumi}.}
\setcounter{enumi}{7}
\tightlist
\item
  Finalmente, para terminar este ejercicio haremos una operación muy
  simple para verificar el buen funcionamiento del programa, escribimos
  en la siguiente celda 5+8 y ejecutamos nuevamente con
  {[}Shift{]}+{[}Enter{]}:
\end{enumerate}

    \hypertarget{markdown}{%
\subsection{Markdown}\label{markdown}}

Markdown es un lenguaje de marcado ligero escrito por John Gruber
(https://daringfireball.net/projects/markdown/) que permite convertir
texto plano en HTML. Aunque suene muy complicado es un lenguaje muy
senciullo de utilizar y está inspirado en la forma de escritura de los
correos electrónicos.

Veremos a continuación algunas convenciones sencillas de este lenguaje y
las aplicaremos en nuestro cuaderno guía:

 \emph{Tomado de:
(https://www.math.ubc.ca/\textasciitilde{}pwalls/math-python/jupyter/markdown/)}

    \hypertarget{salidas-de-texto}{%
\subsubsection{Salidas de texto}\label{salidas-de-texto}}

En la siguiente tabla resumimos algunas convenciones importantes para la
escritura de documentos en Markdown:

\begin{longtable}[]{@{}cc@{}}
\toprule
\textbf{Escrituta en Celda} & \textbf{Salida}\tabularnewline
\midrule
\endhead
\texttt{Escritura\ normal} & Escritura normal\tabularnewline
\texttt{*énfasis*} & \emph{énfasis}\tabularnewline
\texttt{**\ texto\ fuerte\ **} & ** texto fuerte **\tabularnewline
\texttt{código} & \texttt{código}\tabularnewline
\bottomrule
\end{longtable}

    \hypertarget{creaciuxf3n-de-listas}{%
\subsubsection{Creación de listas}\label{creaciuxf3n-de-listas}}

Tambien podemos crear listas, tanto numeradas como no numeradas:

\textbf{Numeradas}:

\begin{verbatim}
     ``` 
     1. Primer elemento
         1. Primero del primero
     2. Segundo elemento
     3. Tercer elemento
     ```
\end{verbatim}

\begin{enumerate}
\def\labelenumi{\arabic{enumi}.}
\tightlist
\item
  Primer elemento

  \begin{enumerate}
  \def\labelenumii{\arabic{enumii}.}
  \tightlist
  \item
    Primero del primero
  \end{enumerate}
\item
  Segundo elemento
\item
  Tercer elemento
\end{enumerate}

\textbf{NO Numeradas}:

\begin{verbatim}
      ``` 
    * Primer elemento
        *otro elemento
    * Segundo elemento
    * Tercer elemento
     ```
\end{verbatim}

\begin{itemize}
\tightlist
\item
  Primer elemento

  \begin{itemize}
  \tightlist
  \item
    otro elemento
  \end{itemize}
\item
  Segundo elemento
\item
  Tercer elemento
\end{itemize}

    \hypertarget{links-e-imuxe1genes}{%
\subsubsection{Links e imágenes}\label{links-e-imuxe1genes}}

Finalmente, usando este lenguaje también podemos conectar links y
visualizar imágenes:

\textbf{Links}

\texttt{{[}Universidad\ Externado{]}\ (https://www.uexternado.edu.co/)}

\href{https://www.uexternado.edu.co/}{Universidad Externado}

\textbf{Imágenes}

\texttt{!{[}Jupyter\ logo{]}(http://jupyter.org/assets/nav\_logo.svg)}
\includegraphics{http://jupyter.org/assets/nav_logo.svg}

    \hypertarget{creaciuxf3n-de-encabezados-en-el-documento}{%
\subsubsection{Creación de encabezados en el
documento}\label{creaciuxf3n-de-encabezados-en-el-documento}}

Es muy útil seccionar el documento por capítulos, subcapítulos,
secciones, subsecciones, entre otros. A continuación, visualizamos las
formas de crear encabezados en cuadernos de jupyter:

\begin{itemize}
\item
  \textbf{Encabezado 1} Escribimos: \texttt{\#\ Encabezado\ 1}
\item
  \textbf{Encabezado 2} Escribimos: \texttt{\#\#\ Encabezado\ 2}
\item
  \textbf{Encabezado 3} Escribimos: \texttt{\#\#\#\ Encabezado\ 3}
\item
  \textbf{Encabezado 4} Escribimos: \texttt{\#\#\#\#\ Encabezado\ 4}
\item
  \textbf{Encabezado 5} Escribimos: \texttt{\#\#\#\#\#\ Encabezado\ 5}
\item
  \textbf{Encabezado 6} Escribimos: \texttt{\#\#\#\#\#\#\ Encabezado\ 6}
\end{itemize}

    \hypertarget{instalaciuxf3n-y-configuraciuxf3n-de-extensiones-para-jupyter}{%
\subsection{Instalación y configuración de extensiones para
Jupyter}\label{instalaciuxf3n-y-configuraciuxf3n-de-extensiones-para-jupyter}}

Las extensiones de Jupyter nos harán la vida más fácil, algunas
extensiones completaran código y nos recordaran algunas funciones que
pueden llegar a ser esquivas para nuestra memoria. Así mismo tendremos
vistas previas de nuestras celdas de texto, entre otras utilidades.

    \hypertarget{instalaciuxf3n}{%
\subsubsection{Instalación}\label{instalaciuxf3n}}

Posterior a la instalación de Anaconda vale la pena instalar algunas
extensiones útiles del notebook de Jupyter. Para esto cerramos todos los
cuadernos activos y reiniciamos Anaconda y Jupyter.

Nuevamente, en la página de inicio de Jupyter:

Seleccionamos \texttt{New} y luego \texttt{Terminal}:

Aparecerá la siguiente ventana:

Escribimos en la consola lo siguiente:

\texttt{pip\ install\ jupyter\_contrib\_nbextensions}

Se instalará el módulo de extensiones de Jupyter. Cuando termine la
instalación escribimos:

\texttt{jupyter\ contrib\ nbextension\ install}

Reiniciamos Jupyter y Anaconda.

    Al abrir nuevamente Jupyter encontraremos algunas cosas nuevas, en
principio una pestaña llamada \texttt{NbExtensions}:

Al ingresar llegamos a la siguiente página, quitamos el chulo del
marcador indicado:

Finalmente seleccionamos las extensiones útiles para nuestros trabajos.
En amarillo las sugeridas:

    Las anteriores extensiones permiten una mejor dinámica entre creador y
usuario. Se sugiere que exploren algunas otras extensiones y verifiquen
los cambios en el cuaderno.


    % Add a bibliography block to the postdoc
    
    
    
\end{document}
